\newpage
\thispagestyle{empty}

\markboth{ \mbox{ACASO E  NECESSIDADE} }{ \mbox{ACASO E NECESSIDADE} }

\noindent {\huge \bf 1 }

\vspace{1cm}
\noindent {\huge \bf Acaso e necessidade}

\addcontentsline{toc}{chapter}{1 \quad Acaso e necessidade}


\setcounter{chapter}{1}
\setcounter{equation}{0}

\label{Intro}

%\newpage
%\thispagestyle{empty}
%\chapter{A aleatoriedade dos fen\^{o}menos}
%\label{Intro}

%\pagenumbering{arabic}

\begin{flushright}
\begin{minipage}{6.5cm}
{\small
\baselineskip=8.5pt
{\it A incerteza \'{e} inevit\'{a}vel no c\'{a}lculo de resultados obser\-va\-cio\-nais. Em geral,  a teoria nos permite calcular somente a probabilidade de obtermos um resultado particular.}

\smallskip
\hfill P.~A.~M.~Dirac
}
\end{minipage}
\end{flushright}



\section{Aleatoriedade dos fen\^{o}menos} \index{fen\^{o}menos aleat\'{o}rios} \index{aleatoriedade dos fen\^{o}menos}

\paragraph*{}
Os fen\^{o}menos naturais podem ser classificados, quanto a possibilidade de ocorr\^{e}ncia, como determin\'{\i}sticos ou aleat\'{o}rios. Se os efeitos associados  a um fen\^{o}meno, devido a determinadas influ\^{e}ncias,  s\~{a}o inequivocamente previs\'{\i}veis, diz-se que os processos envolvidos s\~{a}o determin\'{\i}sticos. Por outro lado, se os efeitos associados a um fen\^{o}meno n\~{a}o s\~{a}o exatamente previs\'{\i}veis,
%mas podem ser associados a certas expectativas relativas de ocorr\^{e}ncia,
os processos envolvidos s\~{a}o ditos aleat\'{o}rios.

Segundo a vis\~{a}o cl\'{a}ssica da Ci\^{e}ncia, a n\~{a}o previsibilidade dos efeitos associados a um fen\^{o}meno estava associada a processos complexos que envolviam a intera\c{c}\~{a}o de um grande n\'{u}mero de sistemas simples.
Assim, o conceito de aleatoriedade estaria vinculado ao comportamento coletivo das mol\'{e}culas de um g\'{a}s, ou \`{a} enorme quantidade de n\'{u}cleos que participam do fen\^{o}meno da radioatividade.

Uma vez que a teoria fundamental da F\'{\i}sica Cl\'{a}ssica -- a Mec\^{a}nica de Newton -- descreve a evolu\c{c}\~{a}o temporal
  de um sistema composto de  part\'{\i}culas por equa\c{c}\~{o}es diferenciais ordin\'{a}rias, pressupunha-se que seu comportamento seria completamente determinado por sua condi\c{c}\~{a}o  inicial.\footnote{Caracterizada pelas posi\c{c}\~{o}es e velocidades iniciais de suas part\'{\i}culas constituintes.}  A aleatoriedade e o acaso em um fen\^{o}meno, ou em um experimento, eram atribu\'{\i}dos \`{a} incapacidade do observador em determinar as condi\c{c}\~{o}es iniciais, ou \`{a} complexidade dos arranjos experimentais.

  Em princ\'{\i}pio,  uma teoria fundamental deveria ser determin\'{\i}stica, tal que  uma dada condi\c{c}\~{a}o inicial estaria associada a um \'{u}nico resultado para a evolu\c{c}\~{a}o de um sistema f\'{\i}sico.
  Assim, teorias probabil\'{\i}sticas n\~{a}o seriam fundamentais, uma vez que poderiam admitir v\'{a}rios poss\'{\i}veis resultados para a evolu\c{c}\~{a}o de um sistema, ao associar expectativas distintas a cada um desses poss\'{\i}veis resultados.\footnote{Sabe-se hoje que, mesmo para sistemas com um pequeno n\'{u}mero de part\'{\i}culas, descritos por teorias causais, as quais em princ\'{\i}pio seriam determin\'{\i}sticas, pequenas perturba\c{c}\~{o}es iniciais podem dar origem a fen\^{o}menos ca\'{o}ticos n\~{a}o previs\'{\i}veis.}

%\pagebreak
Essa vis\~{a}o dos fen\^{o}menos, h\'{a} mil\^{e}nios arraigada no homem, foi tamb\'{e}m com\-par\-ti\-lha\-da por grandes expoentes do pensamento ocidental:

\begin{center}
\begin{minipage}{10cm}
 ``Nada acontece aleatoriamente; tudo acontece por al\-gu\-ma raz\~{a}o e por necessidade.''

\smallskip
\hfill Leucipo
\end{minipage}
\end{center}


\begin{center}
\begin{minipage}{10cm}
 ``Todos os eventos, mesmo aqueles que por sua ir\-re\-le\-v\^{a}n\-cia parecem n\~{a}o se relacionar \`{a}s grandes leis da natureza, delas constituem uma s\'{e}rie t\~{a}o necess\'{a}ria quanto as revolu\c{c}\~{o}es do Sol.''

\smallskip
\hfill P. S. Laplace
\end{minipage}
\end{center}

\begin{center}
\begin{minipage}{10cm}
 ``Deus n\~{a}o joga dados com o Universo.''

\smallskip
\hfill A. Einstein
\end{minipage}
\end{center}





Com o surgimento da Mec\^{a}nica Qu\^{a}ntica, em 1925, a aleatoriedade passa a ser considerada uma caracter\'{\i}stica intr\'{\i}nseca da evolu\c{c}\~{a}o dos fen\^{o}menos e sistemas f\'{\i}sicos, mesmo aqueles com poucos graus de liberdade.\footnote{Sistemas com um pequeno n\'{u}mero de part\'{\i}culas.}  A Mec\^{a}nica Qu\^{a}ntica estabelece que para cada problema h\'{a} de se calcular uma distribui\c{c}\~{a}o de probabilidades que conter\'{a} as informa\c{c}\~{o}es necess\'{a}rias \`{a} descri\c{c}\~{a}o do fen\^{o}meno estudado.

Segundo a Mec\^{a}nica Qu\^{a}ntica\cite{DIRAC}, os valores\footnote{Os valores de uma grandeza resultantes de um procedimento experimental (medi\c{c}\~{a}o) s\~{a}o denominados {\bf medidas}.} ou medidas poss\'{\i}veis de uma  grandeza f\'{\i}sica est\~{a}o associados a certas distribui\coes\ de probabilidade de ocorr\^encia.\footnote{Al\'{e}m do aspecto como expectativa de ocorr\^{e}ncia, o conceito de probabilidade admite outras interpreta\c{c}\~{o}es que ser\~{a}o apresentadas no Cap.~2.}

 Ainda que as medidas as\-so\-ci\-a\-das a uma grandeza n\~{a}o sejam inteiramente previs\'{\i}veis quando efetuadas sob as mesmas condi\c{c}\~{o}es experimentais,  essas medidas est\~{a}o  restritas a um conjunto de valores  condicionado pelas leis da F\'{\i}sica.
 %Nesse sentido, a aleatoriedade dos fen\^{o}menos significa uma  imprevisibilidade parcial.


De modo geral,  uma teoria probabil\'{\i}stica para o estudo dos fen\^omenos naturais ou dos sistemas f\'{\i}sicos deve prover regras que permitam determinar:
\begin{itemize}
\vspace{-0.2cm}
\item os valores (medidas) poss\'{\i}veis para as  grandezas associadas ao  fen\^omeno ou ao sistema  f\'{\i}sico;
\vspace{-0.2cm}
\item as respectivas pro\-ba\-bi\-li\-da\-des de ocorr\^encia das medidas das  grandezas, ou a distribui\cao\ dessas pro\-ba\-bi\-li\-da\-des.
\end{itemize}


\newpage \ \\
\thispagestyle{empty}

%\newpage
%\vspace*{0.5cm}




