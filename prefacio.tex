%\setcounter{chapter}{-1}
%\cleardoublepage%
%===============================================================
\chapter*{Pref\'{a}cio}
%===============================================================
%\ifpdf\relax\else
%\pagestyle{fancy}
%\fi
\addcontentsline{toc}{chapter}{Pref\'{a}cio}%
\markboth{Pref\'{a}cio}{Pref\'{a}cio}%

\baselineskip=14pt

\vspace*{-0.7cm}
\paragraph*{}
Este livro \'{e} fruto da experi\^{e}ncia do autor em an\'{a}lise de dados em experimentos de F\'{\i}sica, desde  sua forma\c{c}\~{a}o inicial nos anos de 1980, na \'{a}rea de raios X e radia\c{c}\~{a}o s\'{\i}ncrotron, nos laborat\'{o}rios  da faculdade de engenharia f\'{\i}sica da Universidade de T\'{o}quio e na {\it Photon Factory} do KEK,\footnote{Acelerador  S\'{\i}ncrotron do  Laborat\'{o}rio de Altas Energias do Jap\~{a}o, em Tsukuba.} at\'{e} a sua participa\c{c}\~{a}o em  grandes experimentos de Altas Energias; DZero  do Fermilab\footnote{Fermi National Accelerator Laboratory, Batavia, Illinois, USA.}  e CMS do CERN,\footnote{Centro Europeu de F\'{\i}sica de Part\'{\i}culas, Genebra, Su\'{\i}\c{c}a.} nos anos de 1991 at\'{e} 2013.

 O material selecionado \'{e} abordado no corpo do texto de modo intuitivo, es\-pe\-ran\-do-se que o leitor tenha familiaridade apenas com os t\'{o}picos b\'{a}sicos de um curso de C\'{a}lculo Diferencial e Integral,  e com alguns conceitos da Estat\'{\i}stica,  todos temas abordados regularmente nos ciclos b\'{a}sicos dos cursos de F\'{\i}sica, Qu\'{\i}mica  e Engenharias.

O texto que serviu de ponto de partida do livro corresponde \`{a}s notas de aula da disciplina de Tratamento estat\'{\i}stico de dados em F\'{\i}sica, mi\-nistrada pelo autor em diversos per\'{\i}odos, ao longo dos \'{u}ltimos 10 anos, na gradua\c{c}\~{a}o e no programa de p\'{o}s-gradua\c{c}\~{a}o do Instituto de F\'{\i}sica Armando Dias Tavares da  Universidade do Estado do Rio de Janeiro (Uerj).
	
Apesar de muitos exemplos serem tomados da \'{a}rea de Altas Energias, as ferramentas e os  m\'{e}todos estat\'{\i}sticos apresentados podem ser utilizados em qualquer experimento que envolva processos aleat\'{o}rios.

O autor agradece aos colegas do Departamento de F\'{\i}sica de Altas Energias da Uerj, Alberto Franco de S\'{a} Santoro, Francisco Caruso, Jos\'{e} Roberto Mahon e Dilson de Jesus Dami\~{a}o, pela leitura, cr\'{\i}ticas  e sugest\~{o}es ao texto.

Um agradecimento especial \`{a} Stella Maris Amadei pela cuidadosa  leitura de diferentes vers\~{o}es de todos os  cap\'{\i}tulos e pelas sugest\~{o}es de estilo.


\vspace*{0.2cm}

%\leftline{\textit{Rio de Janeiro}}

%\vspace*{0.3cm}
\rightline{\textbf{V.O.}}
%\clearpage
%\par
%\vfill

\newpage \ \\
\thispagestyle{empty}


