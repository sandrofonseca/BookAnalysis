\newpage
\
\thispagestyle{empty}

\newpage
\
\thispagestyle{empty}

 \appendix

\markboth{ \mbox{AP\^{E}NDICE A} }{ \mbox{AP\^{E}NDICE A} }

%\noindent {\huge \bf Ap\^{e}ndices}
\

\vspace*{1.0cm}

\noindent {\huge \bf Ap\^{e}ndice A}

\label{topico}
\addcontentsline{toc}{chapter}{A \quad T\'{o}picos sobre probabilidades}


\setcounter{chapter}{1}

\setcounter{equation}{0}


\begin{flushright}
\begin{minipage}{6.5cm}
{\small
\baselineskip=8pt
{\it As loterias s\~{a}o formas de taxa\c{c}\~{a}o (aceitas livremente) das camadas menos privilegiadas da sociedade.}

\smallskip
\hfill D. Ruelle
}
\end{minipage}
\end{flushright}



%\newpage
%\thispagestyle{empty}

%\newpage
%\noindent
%{\bf Ap\^endices}
%\addcontentsline{toc}{section}{Ap\^endices}

% \appendix
% \chapter{T\'opicos sobre probabilidades}

%{\bf A \ \  T\'opicos sobre probabilidades}

%\addcontentsline{toc}{subsection}{I \ \  T\'opicos sobre probabilidades}


%\section{T\'opicos sobre probabilidades}
\label{topicos}

%\def\figurename{\small Fig.~\ref{topicos}.}
%\def\tablename{\small Tab.~\ref{topicos}.}

\section{Eventos equivalentes} \index{eventos! equivalentes}
%$\bullet$ {\bf Eventos equivalentes} :

\paragraph*{}
Se duas vari\'aveis aleat\'orias
$x$ e $y$ est\~ao relacionadas por $y=f(x)$  e $Y$ \'e um conjunto de
condi\coes\ (eventos) sobre a vari\'avel $y$~\footnote{Por exemplo,
uma rela\cao\ do tipo $y < a$, onde $a$ \'e uma constante.}
e $X = \{ x \ | \ f(x) \in Y \}$, diz-se que os conjuntos $X$ e $Y$ s\~ao
pro\-ba\-bi\-lis\-ti\-ca\-men\-te e\-qui\-va\-len\-tes, ou que
s\~ao eventos e\-qui\-va\-len\-tes, no sentido de que
suas probabilidades de ocorr\^encia s\~ao id\^enticas,

\vspace{.5cm}
\centerline{\fbox{\
$P(Y) = P(X)$ \ }}

\subsection*{$\bullet$ distribui\c{c}\~{o}es de probabilidades de eventos equivalentes}

\paragraph*{}
Sendo $f(x)$  a distribui\c{c}\~{a}o de probabilidades ({\bf pdf}) associada \`a vari\'avel $x$  e $y=h(x)$  uma outra vari\'avel aleat\'oria, por exemplo, $y=1+x=h(x)$ e $f(x) =x/2$ para $0 < x < 2$, a fun\cao\ de distribui\cao\  acumulada ({\bf fd})  \ $G(t)$ associada \`a $y$  ser\'a dada por
$$
\begin{array}{ll}
G(t) \! \! \! \! & =P(y\leq  t)  = P(1+x \leq t )  = P(x\leq t-1) \\
   \  \\
     &= \displaystyle \int_0^{t-1} f(x) \ {\rm d}x = \displaystyle \int_0^{t-1} \frac{x}{2} \ {\rm d}x
       = \frac{(t-1)^2}{4}
\end{array}
$$
\noindent e a pdf de $y$ por
$$g(y)= \frac{{\rm d}G}{{\rm d}y} = \frac{y-1}{2} $$

Como
$$ G(t) = P (x \leq \underbrace{t-1}_{h^{-1} (t)} ) =
F [ \underbrace{h^{-1} (t)}_{x(t)} ]$$
$$  \Downarrow $$
$$g(y) = \underbrace{\frac{{\rm d}F}{{\rm d}x}}_{f[x(y)]}
     \frac{{\rm d}(y)}{{\rm d}x}  = \frac{y-1}{2}$$

Admitindo que  $y=h(x)$ seja  uma fun\cao\ mon\'otona, de acordo com as propriedades,

\begin{itemize}
    \item[a)] $f$ crescente
    $$
\begin{array}{ll}
G(t)& =P(y\leq  t )  = P[ f(x)  \leq t ] \\
 & \\
     & = P [ (x\leq h^{-1} (t) ] = F [ h^{-1} (t) ]
\end{array}
$$
$$\Downarrow$$
$$\displaystyle  \underbrace{\frac{{\rm d}G}{{\rm d}y}}_{g(y)} =
  \underbrace{\frac{{\rm d}F}{{\rm d}x}}_{f(x)} \frac{{\rm d}x}{{\rm d}y}
  \ \ \ \ \left( \frac{{\rm d}x}{{\rm d}y} > 0 \right) $$

    \item[b)] $f$ decrescente
    $$
\begin{array}{ll}
G(t)& =P(y\leq t )  =  P [ (x > h^{-1} (t) ]  \\
  & \\
  & = 1 - P [ (x\leq h^{-1} (t) ]
\end{array}
$$
$$\Downarrow$$
$$ \begin{array}{ll}
\displaystyle  g(y) & = \displaystyle  - f(x) \ \frac{{\rm d}x}{{\rm d}y}
 \ \ \ \ \displaystyle  \left( \frac{{\rm d}x}{{\rm d}y} < 0 \right) \\
 & \\
   & \displaystyle  = f(x) \ \left| \frac{{\rm d}x}{{\rm d}y} \right|
   \end{array}
   $$
\end{itemize}

\noindent o   resultado  pode ser sistematizado como

\vspace{.5cm}
\centerline{\fbox{\ $ g(y) =  \displaystyle
f[x(y)] \ \left| \frac{{\rm d}x(y)}{{\rm d}y} \right| $ \ }}




\subsection*{$\bullet$ valor esperado  de fun\c{c}\~{a}o de uma   vari\'avel aleat\'oria}
\index{valor! esperado}
%\addcontentsline{toc}{section}{I Valor m\'edio de fun\cao\ de uma   vari\'avel aleat\'oria}

 \paragraph*{}
   A  determina\cao\  do valor esperado de uma fun\cao\  $y=h(x)$ de uma vari\'avel aleat\'oria $x$ cuja  pdf \'{e}  $f(x)$ \'{e} dada por $$ \langle y \rangle = \int_{-\infty}^{\infty} y \,  g(y) \ {\rm d}y$$
sendo $g(y)$ a pdf  de $y$.

No entanto, uma vez que
    $$
g(y)  = f[x(y)] \ \displaystyle \frac{{\rm d}x}{{\rm d}y} $$
$$   \Downarrow $$
$$
  \langle y \rangle  = \displaystyle
  \int_{-\infty}^{\infty} y \  f[x(y)] \
  \left| \frac{{\rm d}x}{{\rm d}y} \right|  \ {\rm d}y  = \displaystyle
  \int_{-\infty}^{\infty} y \  f[x(y)] \
   \frac{{\rm d}x}{{\rm d}y}  \ {\rm d}y
  $$

\noindent
\vspace{0.3cm}
o valor m\'{e}dio pode ser determinado pela {\bf pdf} de $x$,  por

\vspace{.2cm}
\centerline{\fbox{\ $    \langle h(x) \rangle  =  \displaystyle
  \int_{-\infty}^{\infty}  \  h(x) \
   f(x)   \ {\rm d}x  $ \ }}


